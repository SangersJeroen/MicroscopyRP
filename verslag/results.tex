\section{Results and discussion}

\subsection{Calibration}

The values that have been found for $l_{real}$, $n$, $l_{pixel}$ and the corresponding error are presented in table \ref{table_calibration} for each objective. The images corresponding to each measurement are presented in appendix \ref{appendix_calibration_4}, \ref{appendix_calibration_10} and \ref{appendix_calibration_40}.

%TABLE_CALIBRATION

The values that have been found for $I_max$, $I_min$, $V_{is}$ and the corresponding errors are presented in table \ref{table_vis} in appendix \ref{appendix_vis_table}. In figure \ref{fig_vis_plot}, $V_{is}$is plotted as a function of the spatial frequency for each objective. The images corresponding the measurements are presented in appendix \ref{appendix_vis_4}, \ref{appendix_vis_10} and \ref{appendix_vis_40}.

%plot_visibility

% Analysis

\subsection{Size measurements}

The values that have been found for $d_{hair}$ and $d_{gf}$ are respectively $d_{hair} = XXX $ and $d_{gf} = XXX $. 

% analysis

The images used for this part of the experiment and the values that have been found for $a$, $b$, $A$ and the corresponding errors can be  in appendix \ref{appendix_starch}. A histogram of the values for $A$ is presented in figure \ref{fig_ellipse}.

% particle size histogram

% analysis

\subsection{Birefringence}

The colour planes that were taken into account for this experiment can be seen in figure \ref{fig_bf_planes} in which each Roman numeral corresponds to a colour plane. 

% FIG_BF_PLANES

The values that have been found for $D$, $\delta d$ and the corresponding errors are presented in table \ref{table_bf} in appendix \ref{appendix_table_bf}.

In figure \ref{fig_bf_plot}, $\delta d$ is plotted as a function of $D$. 

%BF PLOT

It follows from the orthogonal distance regression that $\delta n = XXX$. 

%ANALYSIS















\begin{comment}
In theResultaten en discussie (Results and discussion) chapter, you present your results, generally in the form of graphs, and you discuss them. A single small table (m aximum ~10 rows x ~5 columns) is acceptable, but large tables should be in an appendix. In deviation from what many students believe, it is not desirable to separate the presentation and the discussion of results from each other. In professional literature, this is most often done together.\\ •You should introduce each graph:\\ -Why has this graph been included in the report. (What do we want to learn from this graph?).\\-Why   have   you   plotted   this   Y-axis   variable   as   a   function   of   this   X-axis   variable   (which   theoretical/expected relationship is tested/demonstrated in this this graph? \\•Then you tell the reader what (according to you) he/she should see in the graph, limiting yourself toconclusions  that  are  relatively  indisputable.  The  more  speculative  conclusions  should  be  in the  next  chapter.
\end{comment}