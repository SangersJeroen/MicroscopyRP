\section{Discussie}
Met de resultaten van dit experiment is de elasticiteitsmodulus van PVC bepaald. De resulataten waren verkregen door PVC staven van bekende afmetingen in een testopstelling te plaatsen waar de staven aan een kant waren ingeklemd en aan de andere kant steeds werden aangeslagen door de lucht uit een speaker. Doordat de speaker was aangesloten op een functiegenerator was het mogelijk zelf de frequentie van de luchtstoten te bepalen. De elastisciteitmodulus van de langere staaf was bepaald als $E=3.1\pm0.4 GPa$ en van de kleinere staaf was $E=3.6\pm0.3 GPa$. De literatuurwaarde van de elasticiteitsmodulus voor PVC is $E=3.4 GPa$ dit maakt beide waardes niet strijdig met de literatuurwaarde. De grootste fout in het bepalen van de waarde voor de elasticiteitsmodulus komt voort vanuit de frequentie daarom is het belangrijk dat deze nauwkeurig bepaald wordt. Omdat er tijdens het practicum meerdere groepjes tegelijkertijd de waarde van PVC probeerde te bepalen was er hoorbare interferentie van het geluid. Dit had mogelijk ook invloed op de resultaten van de metingen. Het meten van de lengte van de staaf was ook erg foutgevoelig aangezien deze alleen met een simpele liniaal te meten was.