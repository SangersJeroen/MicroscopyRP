\section{Introduction}
Microscopes are used extensively in natural sciences. They enable us to image small objects and structures which cannot be resolved by the human eye. The use of microscopes, could for example, aid in studies of biological cells, molecular structures or object classification. To correctly conduct such microscopy studies, it is vital to know what the possibilities and limits are using a particular microscope are in combination with image improvement techniques.\\
This experiment will focus on a Leica DM EP polarising microscope in combination with a colour CCD and to what extent this set-up can be used to measure the size of small objects and find the birefringence of an unknown crystal. Furthermore, the possibilities of digital image improvement will be investigated.\\
After calibrating the pixels and finding the resolving power with the aid of a microscopic ruler and resolution target, images are made of a human hair, an optical fibre, starch particles, an unknown birefringent crystal and a fungus sample.\\
To find the size of the human hair, fibre and particles, computer techniques will be used. The birefringence of the crystal will be determined by focussing on the differently coloured planes of the crystal. Using this to find the thickness of each colour plane and its colour, the birefringence can be calculated. Finally some python algorithms are implemented on the image of the fungus to investigate improvements on contrast and noise reduction. These algorithms are a sigmoid function and the bilateral mean, contrast enhancing and morphological contrast enhancing filters from the rank Sci-kit package. \\
In section 2 the theory regarding the experiment will be described, followed by the experimental method in section 3. The results and discussion can be found in section 4. Lastly the conclusions in section 5.









\begin{comment}
    TheInleiding ( Introduction) describes:   -The  research  question.  (Be  as  precise  as  possible.  Not:  “We  investigate  on  which  parameters  the  bubble  behaviour  depends”,  but  “we  study  the  relationship  between  the  path  of  bubbles  at  a  microfluidic T-junction, and the velocity and length of those bubbles).\\-The relevance of the research question (for science and/or technology).\\-The state-of-art: what is already know? (including references to prior literature).\\-A brief description of the research method/approach.\\-A brief description of the structure/contents of the rest of the report.\\The  introduction  should  be  self-contained,  without  reference  to  the  manual  or  to  the  remainder  of  the  report, and should be understandable to readers who know nothing about the research.
\end{comment}