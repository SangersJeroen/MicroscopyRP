\section*{Abstract}
In this report the reader will be informed about the experiments performed for the Microscopy research project.\\
\\
For the experiments a Leice DM EP polarising microscope in combination with a CCD camera was used to determine the magnification of certain object lenses, the width of a human hair and optical fibre, the resolving power for multiple different numerical aperatures, the cross section of elliptical starch particles and the birefringence of an unknown crystal. Finally, an image of a biological sample will be improved by using different python algorithms.\\
The magnification and resolving power of the object lenses, the width of a human hair and optical fibre and the cross section of starch particles were found by making images using the CCD camera and counting pixels. This yielded the following results: For the magnification we found pixel lengths equal to $1.5\cdot10^{-6}$, $6.4\cdot10^{-7}$ and $1.6\cdot10^{-7}$ for the $4\times$, $10\times$ and $40\times$ objectives respectively. For the width of a human hair and optical fibre; diameters of $d_{hair}=6.6\pm0.1\cdot10^{-5}$ and $d_{fibre} = 1.26\pm0.01\cdot10^{-4}$ meters were found. The mean cross section of elliptical starch particles was found to be $A_{starch}=1.9\cdot10^{-10}$ $m^2$ with a standard deviation $\sigma = 8 \cdot10^{-11}$ $m^2$. The size measurements yielded results that matched literature. However, the measurements on the starch particles turned out to be inadequate given the shape and number of unresolvable particles.\\
Determining the birefringence of the crystal was done by placing the crystal between to polarising filters and then measuring the difference in height between two colour planes, using this thickness and the colour of the plane, a birefringence number of $\delta n = 5.3\pm0.2\cdot10^{-2}$ was found. Therefore, the crystal could either be astrophylite, silk or piemontite. Given that there is an unidentified measurement error, the validity of the outcome is debatable.\\
Improving the image using the sigmoid function and a contrast enhancing rank filter works well, significantly improving the contrast of the image. A bilateral mean filter proves to be of use when removing noise. It does however, remove some of the details. A morphological contrast enhancement filter can be useful for size measurements or line detection. Combination of the different rank filters also seems to be useful to either reveal much detail or to remove noise, compensated by gain of contrast.

