\section*{Abstract}
In this report the reader will be informed about the experiments performed for the Microscopy research project.\\
For the experiments a Leice DM EP polirising microscope in combination with a CCD camera was used to determine the magnification of certain object lenses, the width of a human hair and optical fiber, the resolving power for multiple different numerical aperatures, the average surface area of an elliptical starch particle and the birefringence of an unkwown crystal. An image of a biological sample will also tried to be improved by using different algorithms.\\
The magnification and resolving power of the object lenses, the width of a humain hair and optical fiber and the average surface area of a starch particle were found by making images using the CCD camera and counting pixels. This yielded the following results: For the magnification we found  pixel lengths equal to $1.5\cdot10^{-6}$, $6.4\cdot10^{-7}$ and $1.6\cdot10^{-7}$ for the $4\times$, $10\times$ and $40\times$ magnification respectively. For the width of a humain hair and optical fiber; diameters of $d_{hair}=6.6\pm0.1\cdot10^{-5}$ and $d_{fiber} = 1.26\pm0.01\cdot10^{-4}$ meters were found. The average area of an elliptical starch particle was found to be $A_{starch}=1.5\cdot10^{-10}$ $m^2$.\\
Determining the birefringence of the crystal was done by placing the crystal between to polarising filters and then measuring the difference in height between two crystal layers, using this thickness and the different colours shown by the crystal a birefringence number of $\delta n = 5.3\pm0.2\cdot10^{-2}$ was found.\\
The image of the biological sample was greatly improved using the implemented algorithm and the $BILAT$, $CONT$ and $MORPH$ $RANK$-filters.\\