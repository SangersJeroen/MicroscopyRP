\section{Conclusions}

The expiremental set-up made the pixel calibration straightforward, the NI Vision software greatly increased the productivity of the workflow. Pictures of the microscopic ruler were easily translated to magnification measurment by the use of edge detection and automatic pixel counting software. Using this method we found the length corresponding to a pixel $l_{pixel}$ to be $1.5\cdot10^{-6}$, $6.4\cdot10^{-7}$ and $1.6\cdot10^{-7}$ for the $4\times$, $10\times$ and $40\times$ magnification respectively.\\
NI Vision also proved to be helpfull with determining the resolving power of the multiple magni\-fication objects by being able to export a linetrace of the pixel intensity, this linetrace was easily readable and could thus be plotted in a graph to find the highest and lowest intensity pixel value for each given bar structure from the USAF target. These high and low values allowed for the calculation of our Visibility. The results were as expected apart from the fact that the highest magnification objective had a lower visibility starting point, this however was explaiend by the fact that this object had an overall lower intensity image because its low numerical aperature meant it cought less light.\\
The images of the human hair and optical fiber were made by focussing them with the smallest magnification first and then, when they were in frame and sharp, switching to a higher magnification setup. For the starch particles the same method was used. Both yielded results that matched current literature. For the human hair diameter we found $d_{hair}=6.6\pm0.1\cdot10^{-5} m$ and for the glass fiber, $d_{fiber} = 1.26\pm0.01\cdot10^{-4}$. For the starch particles a mean value of $A_{starch}=1.5\cdot10^{-10} m^2$.\\
Determining the birefringence of the crystal proved quite a bit more difficult than the previous tasks. We found that the knob to adjust the height of the sample had a bit of play, meaning that sizing the layers was hard and thus our results for the layer thicknesses were unreliable at best. This means that our birefringence value has quite a lot of error. Through linear regression a value for the birefringence of $5.3\pm0.2\cdot10^{-2}$.\\
Improving the image using the sigmoid function and $RANK$ filters worked well, using them we could greatly improve the contrast of the image. The $BILAT$ rank filter was of little use since it removed noise as well as the fine detail one might be interested in.\\

