\section{Conclusions}

The values that were found  for the length corresponding to a pixel $l_{pixel}$ to be $1.5\cdot10^{-6}$, $6.4\cdot10^{-7}$ and $1.6\cdot10^{-7}$ for the $4\times$, $10\times$ and $40\times$ objectives respectively. This was easily achieved with the aid of the NI Vision software.\\
The results for the visibility for the objectives were as expected apart from the fact that the highest magnification objective had a lower visibility starting point, this however was explained by the fact that this object had an overall lower intensity image because its low numerical aperture meant it caught less light. The NI Vision also proved to be helpful with determining the resolving power of the multiple magnification objects by being able to export a line trace of the pixel intensity.\\
The values that was found for the diameter of a human hair and glass fibre are respectively $d_{hair}=6.57\pm0.08\cdot10^{-5} m$ and $d_{gf}=1.26\pm0.01\cdot10^{-4} m$. For the starch particles a mean  value of $A=1.9\cdot10^{-10} m^2$ with a standard deviation of $\sigma = 8 \cdot10^{-11}$ $m^2$ was found for the cross section. The size measurements yielded results that matched literature. However, the measurements on the starch particles turned out to be inadequate given the shape and number of unresolvable particles.\\
The value that was found for the birefringence of the crystal is $\Delta n = 5.3\pm0.2\cdot10^{-2}$. Therefore, the crystal could either be astrophylite, silk or piemontite. Given that there is an unidentified measurement error, the validity of the outcome is debatable.\\
Improving the image using the sigmoid function and a contrast enhancing rank filter works well, significantly improving the contrast of the image. A bilateral mean filter proves to be of use when removing noise. It does however, remove some of the details. A morphological contrast enhancement filter can be useful for size measurements or line detection. Combination of the different rank filters also seems to be useful to either reveal much detail or to remove noise, compensated by gain of contrast.  \\


